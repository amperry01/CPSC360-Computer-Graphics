\documentclass[12pt]{article}
\usepackage{amsmath, amssymb}
\usepackage{tikz}
\usepackage[margin=1in]{geometry}
\usepackage{graphicx}

\begin{document}

\noindent {\large \textbf{Assignment 8 - OpenGL for 3D Viewing 1 \& 2}} \par
\medskip
\noindent {\large Alyssa Perry} \par
\medskip
\noindent {\large Due: Wednesday, April 16, 2025} \par
\medskip
\noindent {\large Note: Using 1 extra day} \par
\medbreak

\section*{Question 1}
To set up the camera in OpenGL, we use the \texttt{gluLookAt} function:
\begin{verbatim}
gluLookAt(-10, 0, 0, 1, 0, 0, 0, 1, 0)
\end{verbatim}

\noindent{Explanation:}
\begin{itemize}
    \item Eye (camera) position: $e = (-10, 0, 0)$
    \item Look-at point: $(1, 0, 0)$
    \item View-up vector: $(0, 1, 0)$
\end{itemize}
This places the camera along the negative x-axis, looking toward the origin and aligned upward with the y-axis.

\begin{center}
\begin{tikzpicture}[scale=1.2]
  \draw[->] (0,0) -- (4,0) node[right] {$x$};
  \draw[->] (0,0) -- (0,3) node[above] {$y$};
  \draw[->] (0,0) -- (-1,-1) node[below] {$z$};

  \node at (-4.5,0.25) {Camera Position};
  \filldraw[black] (-3,0) circle (2pt) node[below] {$e$ (-10,0,0)};

  \draw[->, thick] (-3,0) -- (1,0) node[below right] {gaze/look-at (1,0,0)};
  \draw[->, thick, blue] (-3,0) -- (-3,1) node[above, left] {view up (0,1,0)};
\end{tikzpicture}
\end{center}

\section*{Question 2}
To calculate the world space coordinates of the near and far planes, use the viewing direction to find world space coordinates of the near and far planes. 
The near and far planes are defined by the distances from the eye position along the viewing direction.
Determine which coordinate axes are perpendicular to the planes.
\medskip

\noindent{Given:}
\begin{itemize}
    \item Eye position: $e = (-10, 0, 0)$
    \item Look-at point: $a = (1, 0, 0)$
    \item Near distance: $-10$ (behind the camera)
    \item Far distance: $70$ (in front of the camera)
\end{itemize}

\noindent{Calculate the near plane coordinates:}
\[
\text{near plane} = e + \text{near distance} \cdot \text{gaze direction} 
\]
\[
= (-10, 0, 0) + (-10) \cdot (1, 0, 0) = (-20, 0, 0)
\]

\bigskip
\noindent{Calculate the far plane coordinates:}
\[
\text{far plane} = e + \text{far distance} \cdot \text{gaze direction} 
\]
\[
= (-10, 0, 0) + (70) \cdot (1, 0, 0) = (60, 0, 0)
\]

\bigskip
\noindent{The near and far planes are defined by the following equations:}
\[
\text{near plane: } x = -20 \quad \text{and} \quad \text{far plane: } x = 60
\]

\bigskip
\noindent{The planes are perpendicular to the x-axis.}
\medskip
\noindent{To summarize:}
\begin{itemize}
    \item Near plane coordinates: $(-20, 0, 0)$
    \item Far plane coordinates: $(60, 0, 0)$
    \item Near plane equation: $x = -20$
    \item Far plane equation: $x = 60$
    \item Distance from eye to near plane: $10$
    \item Distance from eye to far plane: $70$
\end{itemize}

\bigskip
\noindent{The planes are perpendicular to the x-axis because the gaze direction is along the x-axis.}

\section*{Question 3}
Given:
\begin{itemize}
  \item Eye position: $(-10, 0, 0)$
  \item Near plane at $x = -9$
  \item Far plane at $x = 10$
  \item Clipping window size: $20 \times 10$ (width $\times$ height)
\end{itemize}

\noindent{Aspect ratio:}
\[
\text{Aspect ratio} = \frac{\text{width}}{\text{height}} = \frac{20}{10} = 2.0
\]
\noindent{Field of view:}
\[
\tan\left(\frac{\theta}{2}\right) = \frac{y_{top}}{z_{near}} = \frac{5}{1}
\]
\[
\theta = 2 \cdot \tan^{-1}(5) \approx 157.38^\circ
\]
\noindent{Near and far distances from the eye:}
\[
\text{Near distance} = 1 \quad \text{and} \quad \text{Far distance} = 20
\]
\noindent{OpenGL function:}
\begin{verbatim}
gluPerspective(157.38, 2.0, 1, 20)
\end{verbatim}

\noindent{This uses the vertical field of view and correct positive distances from the camera to the clipping planes, consistent with how OpenGL and our lecture slides define the projection setup.}

\section*{Question 4}
\noindent{Given:}
\begin{itemize}
  \item Display window size: $500 \times 500$ (width $\times$ height)
  \item Viewport size: $200 \times 200$
  \item The viewport is centered within the display window
\end{itemize}

\noindent{Explanation:}
\medskip

\noindent{The \texttt{glViewport} function defines the rectangular area of the window (in pixels) where the final image will be mapped. It takes four parameters: the x and y coordinates of the lower-left corner of the viewport, and the width and height of the viewport.}

\begin{itemize}
    \item The display window is $500 \times 500$, so the full range of pixel coordinates goes from (0, 0) at the bottom-left to (500, 500) at the top-right.
    \item The viewport is $200 \times 200$ and must be centered. To center it, we calculate the x and y offset from the display window origin as: 
    \[
    x = \frac{500 - 200}{2} = 150, \quad y = \frac{500 - 200}{2} = 150
    \]
    \item Therefore, the bottom-left corner of the viewport is at (150, 150).
\end{itemize}

\noindent{This ensures the viewport sits exactly in the middle of the window and scales rendering to a smaller 200 by 200 region while preserving its position.}

\noindent{OpenGL function:}
\begin{verbatim}
glViewport(150, 150, 200, 200)
\end{verbatim}

\end{document}
