\documentclass{article}
\usepackage{amsmath}
\usepackage{graphicx}

\begin{document}

\noindent {\large \textbf{Assignment 6 - Viewing I \& II}} \par
\medskip
\noindent {\large Alyssa Perry} \par
\medskip
\noindent {\large Tuesday, April 1, 2025} \par
\medbreak

\begin{enumerate}
    \item Screen Space Description
    \medbreak
            
    You can think of screen space as the 2D grid where everything gets drawn on your screen. It's kind of like a canvas that your computer uses to show stuff. 
    The origin usually starts in a corner (like the top-left), and the x and y axes go across and down. It’s often a left-handed system, which means the axes follow a certain direction convention. 
    It’s useful because it makes positioning things on screen a lot more predictable.

        \medbreak
        To transform a 3D point to screen space using perspective projection:
        \begin{enumerate}
            \item \textbf{Model Transformation}: Convert object coordinates to world space.
            \item \textbf{View Transformation}: Convert world coordinates to camera/view space.
            \item \textbf{Projection Transformation}: Apply a perspective projection matrix to get clip space coordinates.
            \item \textbf{Clipping}: Remove any parts of the geometry outside the view frustum.
            \item \textbf{Perspective Divide}: Divide by $w$ to get normalized device coordinates (NDC).
            \item \textbf{Viewport Transformation}: Convert NDC to screen space using screen width and height.
        \end{enumerate}

    \item Perspective View Volume Properties
        \medbreak
        \textbf{Given:} $z_{\text{near}} = -1$, distance to far = 49 $\Rightarrow z_{\text{far}} = -50$

        \textbf{Near clipping window:} width = 200, height = 100

        \textbf{a. Near and Far Planes:} $z_{\text{near}} = -1$, $z_{\text{far}} = -50$ 

        \textbf{b. Clipping Window:} $x_l = -100,\quad x_r = 100,\quad y_b = -50,\quad y_t = 50$

        \textbf{c. Perspective to Orthographic Matrix $M_{\text{persp}\to\text{ortho}}$:}
        \[
        M_{\text{persp}\to\text{ortho}} =
        \begin{bmatrix}
        -1 & 0 & 0 & 0 \\
        0 & -1 & 0 & 0 \\
        0 & 0 & -51 & -50 \\
        0 & 0 & 1 & 0
        \end{bmatrix}
        \]
        \textbf{d. Final Normalized Matrix $M_{\text{persp}\to\text{norm}}$} can be obtained by multiplying with the orthographic normalization matrix.

    \item Viewport Transformation
        \medbreak
        \textbf{a. Viewport Matrix:}
        \[
        M_{\text{viewport}} =
        \begin{bmatrix}
        6 & 0 & 0 & 6 \\
        0 & 5 & 0 & 5 \\
        0 & 0 & 1 & 0 \\
        0 & 0 & 0 & 1
        \end{bmatrix}
        \quad \text{(width = 12, height = 10)}
        \]
        \textbf{b. Transform NDC (0.5, 0.5, 0.5):}
        \[
        \begin{bmatrix}
        x \\
        y \\
        z \\
        1
        \end{bmatrix}
        =
        \begin{bmatrix}
        6 & 0 & 0 & 6 \\
        0 & 5 & 0 & 5 \\
        0 & 0 & 1 & 0 \\
        0 & 0 & 0 & 1
        \end{bmatrix}
        \begin{bmatrix}
        0.5 \\
        0.5 \\
        0.5 \\
        1
        \end{bmatrix}
        =
        \begin{bmatrix}
        9 \\
        7.5 \\
        0.5 \\
        1
        \end{bmatrix}
        \]
        \textbf{c. Pixel Coordinates:} $(x_{\text{pixel}}, y_{\text{pixel}}) = (9, 7.5)$

    \item Calculating Vertical Field of View (FOV)
        \medbreak
        \textbf{Given:} Near plane at $z = -10$, Window size = $10 \times 12$
       
        \textbf{Vertical FOV:}
        \[
        \theta = 2 \cdot \tan^{-1}\left(\frac{H}{2 \cdot |z_{\text{near}}|}\right)
        = 2 \cdot \tan^{-1}\left(\frac{12}{20}\right)
        = 2 \cdot \tan^{-1}(0.6)
        \approx 61.9^\circ
        \]

\end{enumerate}

\end{document}